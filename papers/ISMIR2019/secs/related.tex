\section{Related}

Prior work introduced the problem of DSP-PBE, but was limited to progams that had a fixed structure and only used LPF, HPF, and white noise~\cite{SantolucitoFARM} .
This limitation is too restrictive for the tool to be useful in practice, these types of filters are already captured by room impulse response measuring techniques.
Many interesting DSP program go beyond this functionality.
In addition, in order to be useful as a tool for learning, DSP-PBE must support the synthesis of a code with different structures to expose the new users to various programming constructs.
In our work, we extend DSP-PBE to noncommutative filter by allowing the exploration of new structural forms for the DSP program - achieving a synthesis technique that is more powerful than current techniques.

One way to look at the problem of DSP-PBE is a specialized type of feature extraction, where the target feature is a block of executable code.
There has been significant work in the domain of feature extraction for audio, from low-level~\cite{muller2011chroma}, mid-level~\cite{AljanakiS18}, and high-level~\cite{PandaMP18, mathieu2010yaafe, lidy2005evaluation} features.
It may be possible to correlate these features to particular filter values and structures in order to leverage existing feature extraction work for the purposes of DSP-PBE.

Programming by Example with noisy data has been explored in prior~\cite{raychev2016learning}, but captures a distinct domain compared to DSP-PBE.
In Noisy Programming by Example context, the interpretation of noisy is as anomalous - some examples are outliers and should be disregarded.
This is an appropriate assumption in the context of, for example, in faulty sensors or excel spreadsheets, where a value may fall completely outside the normal range due to an electrical error or a typo.
DSP-PBE could be seen as taking an alternative interpretation of noise - that as small variations from the true value.
In DSP-PBE we do not treat the sample pairs from the input and output audio files as hard constraints on the system - rather we are looking for a program that maps these points as closely as possible to each other.

Further work in the PBE has combined machine learning approaches to learning approximate solutions, such as in the tool RobustFill~\cite{devlin2017robustfill}.
% also see the work of https://arxiv.org/pdf/1611.01855.pdf and https://openreview.net/forum?id=Skp1ESxRZ
In particular, this work finds that traditional PBE systems are more accurate when \textit{all} examples are correct, while their neural network approach is more accurate when \textit{most} examples are correct.
Again, in the domain of DSP-PBE, we must handle the case when \textit{none} of the examples are \textit{correct}, but rather all examples are approximately correct.
RobustFill works by generating a ``latent'' program representation (a neural network), that is then translated into a program.
On our work, we have leveraged machine learning (stochastic gradient descent), in combination with a greedy search algorithm to handle the discrete aspect of a DSP filter (the structure).
There may be fruitful opportunities to combine these approach and apply a neural network-guided approach to DSP-PBE.

Audio resynthesis~\cite{masri1996improved} is way of reconstructing sounds that is similar to DSP-PBE.
Resynthesis builds a synthesizer to recreate a similar sound by decomposing an input sound into its spectrogram and identifying amplitudes at particular frequencies.
In contrast to DSP-PBE, which produces audio filters, resynthesis produces a generative synthesizer to produce sounds without any input.
As a generative synthesizer, the output of resynthesis cannot be used to understand the process by which the original sound was modified, whereas \ourTool allows users to understand the modifications that are applied to a sound in order to create their own programs.
As an example, resynthesis could be used to recreate the sound of a muted trumpet, but \ourTool allows users to understand how to write a program that transforms an unmuted trumpet sound into the muted trumpet.
