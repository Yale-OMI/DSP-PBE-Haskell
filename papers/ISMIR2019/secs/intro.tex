
\section{Introduction}

There has been a proliferation of new programming languages for audio Digital Signal Processing (DSP) including languages such as SuperCollider~\cite{supercollider}, Faust~\cite{orlarey2009faust}, and PureData~\cite{puredata}.
These languages provide a high-level interface to make DSP programming more approachable for new-comers to the field of audio programming.
DSP programming poses a particular challenge to novice programmers, as writing a DSP program requires an understanding of traditional programming concepts such as loops and conditionals, as well as the ability to reason about DSP issues such as time vs frequency domain.
To assist beginning DSP programmers we turn to the Programming by Example paradigm.

Programming by Example (PBE)~\cite{cypher93,lieberman01} is a program synthesis technique 
where users do not write code directly, but instead provide examples that illustrate how the envisioned 
code should behave. A PBE synthesis engine takes those input and output examples and 
automatically generates code that implements the illustrated functionality. Of course, there might be more than
one program that is a good fit for the given examples, so the user may need to provide more examples.
However, the PBE techniques are based on various additional domain knowledge and heuristics, and have even been successfully applied in industrial software, for example in spreadsheets with the FlashFill tool~\cite{flashfill}.

Our goal is to make DSP programming more accessible to a larger audience by using Digital Signal Processing Programming by Example (DSP-PBE)~\cite{SantolucitoFARM}.
In order to do this, we developed a tool called \ourTool for DSP-PBE.
DSP-PBE is a paradigm that uses techniques from programming languages research to generate DSP programs based on audio input and output examples.
\ourTool extends prior work on DSP-PBE~\cite{SantolucitoFARM} with the ability to synthesize nontrivial filter types and generate a range of various code structures.
Additionally, \ourTool generates executable code, in SuperCollider, that allows the user to inspect, modify, and reuse the result of synthesis in their own coding projects.

\ourTool takes input $i$ and output $o$, which are audio files, and synthesizes the DSP program, $F$, that minimizes the distance between the transformed input, $F(i)$, and the output $o$.
A key part of this technique is that the user receives readable DSP program code that can be further tuned or edited as the user sees fit, opening the door for learning opportunities and creative invention.


\subsection{Motivating Example}


\begin{figure}
\begin{lstlisting}
( 
SynthDef(\dsp_pbe, {|out=0|

   var main_in, id7, out6, lpf5, hpf4, psh1;
   main_in = PlayBuf.ar(2, ~buf);
   psh1 = FreqShift.ar(\pitchRatio, -399.999, 
                       \mul, 0.55, 
                       \in, In.ar());
   hpf4 = HPF.ar(\freq, 10100.0, 
                 \mul, 1.562e-10, 
                 \in, psh1);
   lpf5 = LPF.ar(\freq, 3860.002, 
                 \mul, 0.85, 
                 \in, psh1);
   out6 = Mix.ar(2, [hpf4, lpf5]);
   id7 = 0.7 * out6;
   Out.ar(out, id7);

}).add;
)
\end{lstlisting}
\caption{The SuperCollider program synthesized by \ourTool to simulate the effect of a trumpet hat mute.}
\label{fig:sc_code}
\end{figure}

In order to illustrate how \ourTool works, consider a scenario where a user wants apply the effect of a trumpet mute on their own sound, for example a recording of their voice.
The audio files for this example are given in our evaluation (cf. Sec.~\ref{sec:eval}, Table~\ref{table:eval}).
To start, a user provides an audio input example file of the trumpet without a mute (``00 none'' in Table~\ref{table:eval}), and an audio output example file (``01 hat'' in Table~\ref{table:eval}) of the trumpet playing the same note with the mute.
The user then invokes \ourTool on these two files, and the tool generates the SuperCollider program shown in Fig.~\ref{fig:sc_code}.
With this code, the user can then both use the program directly in a larger SuperCollider project, or edit the code directly.

In summary, this paper makes the following contributions.

\begin{enumerate}
\item We propose a framework for the synthesis of DSP programs that utilize noncommutative filter types 
\item We propose an algorithm to search through the possible structural forms of a DSP filter program
\item We present our tool, \ourTool, which synthesizes SuperCollider programs from audio input/output examples, and evaluate \ourTool over a set of benchmarks
\end{enumerate}


