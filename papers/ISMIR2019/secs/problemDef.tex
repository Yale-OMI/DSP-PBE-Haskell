
\section{Problem Statement}

One of the core challenges in applying Programming by Example to DSP (DSP-PBE) is to quantify how close our synthesized solution is to the correct solution.
However, the closeness of two audio files is a subjective measure - we do not necessarily know what features of the input/output audio pair the user is trying to emulate.
We use an audio distance measure from~\cite{SantolucitoFARM} to calculate the distance between two audio files.
This metric approximates a measure of how different two audio clips sound to the human ear in a similar style to audio fingerprinting~\cite{wang2003industrial,fingerprinting}.
To calculate the distance beteen two files, the function takes a sliding window FFT of fixed size, denoted $FFT(x,t)$ for the FFT of audio $x$ at time point $t$, over all time points $ts$.
It then plots the constellation of amplitude and frequency peaks $p$, and takes the euclidean distance between the points of the constellation.

%
\begin{align*}
\distFxn(x,y) =  \sum_{t=0}^{ts}\ \sum_{i=0}^{p} euclid\Big(\ &peak(i,FFT(x,t)), \\ &peak(i,FFT(y,t))\ \Big)
\end{align*}
%

While measuring frequency peaks requires temporally aligned audio, alignment can be achieved by using recent advances in dynamic time warping techniques for audio~\cite{carabias2015audio} and other audio alignment techniques~\cite{ArztL18}.
To contain the scope of the DSP-PBE problem, we assume that the user has provided audio that is temporally aligned.
As a technical note, this also requires that the sample rates of the audio match and the amplitudes of the examples have been normalized.
Again we assume the user handles amplitude normalization and resampling to align sample rates as necessary for simplicity.

In DSP-PBE the goal is to find a DSP filter $F$ to minimize the distance between $F$ applied to an input audio file, $i$, and an output audio file $o$.
The intuition is that if $\distFxn(F(i),o) = 0$, we have successfully mapped the input to the output, i.e. $F(i) = o$.
This property is called the \textit{identity of indiscernibles}~\cite{frechet1906quelques} - given a distance metric $d$, it must hold that $d(x,y) = 0 \Leftrightarrow x = y$.
Unfortunately however, this property does not hold for the \distFxn function that we use.
An easy counterexample to show that $\distFxn$ violates the identity of indiscernibles property is to take two audio files that have identical peaks, but differ in amplitudes at frequencies outside the peaks.
These inputs then violate the property as two audio files have distance zero as calculated by $\distFxn$, but are not identical audio files.
This is something of a pathological example, as in most cases $\distFxn$ is close enough for the purposes of DSP-PBE.
We simply remark that finding an aural distance that obeys the identity of indiscernibles property is a valuable direction for future work 

\subsection{Defining the search space}

In order to formalize the search space of DSP-PBE, we define a grammar of filters, as shown in Fig.~\ref{fig:grammar}.
This grammar defines the tree-like structure of all allowable DSP programs that our synthesis technique is able to find.
The grammar states that we may construct a DSP filter program, $DSP$, by composing programs in one of two ways.
We allow sequential composition with the $\arrComp$ operator, which indicates the two component programs share input, and their output is summed in the time domain.
We also allow parallel composition with the $\parallelCompose$ operator, which indicates that the first component program pipes its output audio to the next component program.
The content at base level of a filter is a \dspnode, which is selected from the set of primitive filters available to synthesis.
This set can be expanded as needed by the user depending on the use case.
We limit the set of \dspnode here to a core set illustrating both commutative ($LFP, HPF, WN$) and noncommutative ($Reverb, PitchShift$) operators.
The app level for each node indicates amplitude scaling of the output.

\begin{figure}
\begin{flalign*}
DSP & := \\
& |\ DSP \arrComp DSP\ \qquad & \text{\color{gray} sequential composition}  \\
& |\ DSP \parallelCompose DSP\ & \text{\color{gray} parallel composition} \\
& |\ \dspnode& \\
DSP&Node :=  & \\
& |\ LPF \ [0,20k] \ [0,1] &\text{\color{gray} threshold, app}\\
& |\ HPF \ [0,20k] \ [0,1] &\text{\color{gray} threshold, app}\\
& |\ Reverb \ [0,1] \ [0,1] \ [0,1] & \text{\color{gray} room size, mix, app}\\
& |\ PitchShift\ [-1,1] \ [0,1] & \text{\color{gray} shift amount, app}\\
& |\ WhiteNoise \ [0,1] &\text{\color{gray} app}
\end{flalign*}
\caption{The grammar of DSP filters we consider includes lowpass filters (LPF), highpass filters (HPF), and can also be extended with other user supplied primitives.}
\label{fig:grammar}
\end{figure}


