\section{Structural Synthesis}
\label{sec:struct}
In order to generate a reduced grammar $G', \languageOf{G'} \subset \languageOf{G}$, we require domain-specific knowledge from the user.
Specifically, the user provides our synthesis system with a set of grammar transformations \mbox{$t: \typeOfGrammar \to \typeOfGrammar$}, where the transformations strictly reduce the size of the set terminals that the grammar can generate, \mbox{$\forall G \in \typeOfGrammar. \ \languageOf{t(G)} \subseteq \languageOf{G}$}.
The user then also provides a refinement predicate $r: \exampleDomain \to \exampleDomain \to Boolean$ that describes when each transformation $t$ can be applied. 

The language containment problem ($\languageOf{G_{1}} \subset \languageOf{G_{2}}$) is important for us here since we want to be sure that we are strictly decreasing the space of candidate programs to ensure that our search terminates.
The problem of checking language containment ($\languageOf{G_{1}} \subset \languageOf{G_{2}}$) for arbitrary context-free grammars $G_1$ and $G_2$ is undecidable in general~\cite{hopcroft1969equivalence}.
Ensuring that arbitrary transformations of the grammar indeed shrinks the search space is then not feasible.
However, our structural refinements do not require this power.
We instead limit the transformation to be one that only reduces the production rules $\productionRules_G$ to the production rules $\productionRules_{t(G)}$.
Specifically, the transformation $t$ is constructed by a mapping of $RHS$ components of the production rules.

\begin{align*}
P_{t(G)} =& \ \{ \ (lhs,\ rhs(lhs)\ )\  \} \\
&\text{where}\\
&\qquad lhs = proj_{LHS}(P_G) \\
&\qquad rhs(l) \subseteq proj_{RHS}(l) 
\end{align*}

The user must specify when each transformation may be applied.
If the input/output examples have some relation $r$, then we can transform the search space in a way that removes potential programs that are not good.
Put formally, we want to define relation $r$ and transformation $t$, such that all the programs that we have removed have a higher cost than some program in the transformed grammar.

\begin{align*}
   r(i,o) &\implies \\
   & \forall p \in \languageOf{G} \setminus \languageOf{t(G)}. \\
   & \exists p' \in \languageOf{t(G)}. \\
   & \distFxn(o,p(i)) > \distFxn(o,p'(i)) 
\end{align*}

\begin{exmp} Limiting the DSP Grammar

%Should this come earlier?
In the grammar of our running example, we have two filters, a low-pass filter, $LFP \ [0,20k]\ [0,1]$, and a high-pass filter $HPF\ [0,20k]\ [0,1]$.
A low-pass filter lets frequencies lower than a given threshold pass through to the output.
Concertizing the threshold constants of $LPF$ at a value of 1500 Hz, and the application constant at 1, we then get $LPF \ 1500 \ 1$.

Applying $LPF \ 1500 \ 1$ to an audio file with a flat frequency response look as follows:

\markk{image of audio here}

We then build a relation from this observation, using a similar syntax to the refinement type system for Haskell, LiquidHaskell~\cite{vazou2014refinement}, 
Based on the above observation, we can write a refinement type on $LPF$ to describe that says the amplitude of the frequencies greater than the threshold frequency have decreased in the output Audio.
%
\begin{align*}
  &\texttt{lpf :: t:\reals} \to  \texttt{xs:Audio} \to\ \texttt{ys:Audio}\ \mid \\
  &\forall f_1 \in  \texttt{spectrogram(xs)}.\ \forall f_2 \in \texttt{spectrogram(ys)}. \\
  &(f_1 > \texttt{t}  \land  f_2 > \texttt{t}  \land f_1 == f_2) \implies \texttt{amp}(f_1) > \texttt{amp}(f_2)
\end{align*}

Where \texttt{t} represents the level at which the lowpass filter is applied, \texttt{spectrogram} represents the spectrogram of the sound sample, $f_i$ represents a frequency, and \texttt{amp()} represents the amplitude of the frequency. 

Similarly, we can build a relation to describe a high-pass filter as a refinement type that says the amplitude of the frequencies less than the threshold frequency have decreased in the output Audio.
%
\begin{align*}
  &\texttt{hpf :: t:\reals} \to\ \texttt{xs:Audio} \to\ \texttt{ys:Audio}\ \mid \\
  &(\forall f_1 \in \texttt{spectrogram(xs)}. \forall f_2 \in \texttt{spectrogram(ys)}). \\
  &(f_1 < \texttt{t} \land f_2 < \texttt{t} \land f_1 == f_2) \implies \texttt{amp(}f_1\texttt{)} > \texttt{amp(} f_2 \texttt{)} 
\end{align*}

Notice that in these refinement types, we only need to calculate the spectrogram for the input and output statically.
As opposed to the current technique of generating filters, applying them, and the calculating the aural distance, this approach is relatively static.
We could quickly check many threshold values over the input and output examples.

\end{exmp}
