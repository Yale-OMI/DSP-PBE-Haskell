%% For double-blind review submission
\documentclass[sigplan,screen]{acmart}\settopmatter{printfolios=true}

\usepackage{etex}
%% For single-blind review submission
%\documentclass[acmlarge,review]{acmart}\settopmatter{printfolios=true}
%% For final camera-ready submission
%\documentclass[acmlarge]{acmart}\settopmatter{}

%% Note: Authors migrating a paper from PACMPL format to traditional
%% SIGPLAN proceedings format should change 'acmlarge' to
%% 'sigplan,10pt'.


%% Some recommended packages.
                        %% http://ctan.org/pkg/booktabs
\usepackage{subcaption} %% For complex figures with subfigures/subcaptions
                        %% http://ctan.org/pkg/subcaption


\newcommand{\cL}{{\cal L}}
\let\terms\undefined



\usepackage{xspace}
\usepackage{caption}
\usepackage{subcaption}


\newcommand{\synthFilter}{\ensuremath{\mathcal{F}}\xspace}



%\makeatletter\if@ACM@journal\makeatother
%% Journal information (used by PACMPL format)
%% Supplied to authors by publisher for camera-ready submission

%% Copyright information
%% Supplied to authors (based on authors' rights management selection;
%% see authors.acm.org) by publisher for camera-ready submission
%\setcopyright{acmcopyright}
%\setcopyright{acmlicensed}
%\setcopyright{rightsretained}
%\copyrightyear{2017}           %% If different from \acmYear


%% Bibliography style
\bibliographystyle{ACM-Reference-Format}
%% Citation style
%% Note: author/year citations are required for papers published as an
%% issue of PACMPL.
\citestyle{acmauthoryear}   %% For author/year citations

\setcopyright{acmcopyright}
\acmPrice{}
\acmDOI{10.1145/3133888}
\acmYear{2017}
\copyrightyear{2017}
\acmJournal{PACMPL}
\acmVolume{1}
\acmNumber{ICFP}
\acmArticle{64}
\acmMonth{10}


\begin{document}

%% Title information
\title[Programming Music by Example]{Programming Music by Example: Synthesizing Digital Signal Processing Programs}
                                        %% [Short Title] is optional;
                                        %% when present, will be used in
                                        %% header instead of Full Title.
%\titlenote{with title note}             %% \titlenote is optional;
                                        %% can be repeated if necessary;
                                        %% contents suppressed with 'anonymous'
%\subtitle{Subtitle}                     %% \subtitle is optional
%\subtitlenote{with subtitle note}       %% \subtitlenote is optional;
                                        %% can be repeated if necessary;
                                        %% contents suppressed with 'anonymous'

%% Author information
%% Contents and number of authors suppressed with 'anonymous'.
%% Each author should be introduced by \author, followed by
%% \authornote (optional), \orcid (optional), \affiliation, and
%% \email.
%% An author may have multiple affiliations and/or emails; repeat the
%% appropriate command.
%% Many elements are not rendered, but should be provided for metadata
%% extraction tools.

%% Author with single affiliation.
\author{Mark Santolucito}
\orcid{0000-0001-8646-4364}             %% \orcid is optional
\affiliation{
  \department{Computer Science}              %% \department is recommended
  \institution{Yale University}            %% \institution is required
  \streetaddress{51 Prospect St.}
  \city{New Haven}
  \state{CT}
  \postcode{06511}
  \country{USA}
}
\email{mark.santolucito@yale.edu}          %% \email is recommended

\author{Kate Rogers}
\affiliation{
  \department{Computer Science}              %% \department is recommended
  \institution{Yale University}            %% \institution is required
  \streetaddress{51 Prospect St.}
  \city{New Haven}
  \state{CT}
  \postcode{06511}
  \country{USA}
}
\email{kate.rogers@yale.edu}          %% \email is recommended


\author{Aedan Lombardo}
\affiliation{
  \department{Computer Science}              %% \department is recommended
  \institution{Yale University}            %% \institution is required
  \streetaddress{51 Prospect St.}
  \city{New Haven}
  \state{CT}
  \postcode{06511}
  \country{USA}
}
\email{aedan.lombardo@yale.edu}          %% \email is recommended


\author{Ruzica Piskac}
\affiliation{
  \department{Computer Science}              %% \department is recommended
  \institution{Yale University}            %% \institution is required
  \streetaddress{51 Prospect St.}
  \city{New Haven}
  \state{CT}
  \postcode{06511}
  \country{USA}
}
\email{ruzica.piskac@yale.edu}          %% \email is recommended


\titlenote{This research was sponsored by the NSF under grants
CCF-1302327 and CCF-1715387.}




%% Paper note
%% The \thanks command may be used to create a "paper note" ---
%% similar to a title note or an author note, but not explicitly
%% associated with a particular element.  It will appear immediately
%% above the permission/copyright statement.
\thanks{Thanks to Thomas Murphy for his guidance in thinking through this problem.}       %% \thanks is optional
                                        %% can be repeated if necesary
                                        %% contents suppressed with 'anonymous'


%% Abstract
%% Note: \begin{abstract}...\end{abstract} environment must come
%% before \maketitle command
\begin{abstract}
Programming by example allows users to create programs without coding, by simply specifying input and output pairs.
We introduce the problem of digital signal processing programming by example (DSP-PBE), where users specify input and output wave files, and a tool automatically synthesizes a program that transforms the input to the output.
This program can then be applied to new wave files, giving users a new way to interact with music and program code.
We formally define the problem of DSP-PBE, and provide a first implementation of a solution that can handle synthesis over commutative filters.
\end{abstract}


%% 2012 ACM Computing Classification System (CSS) concepts
%% Generate at 'http://dl.acm.org/ccs/ccs.cfm'.
\begin{CCSXML}
%<ccs2012>
%<concept>
%<concept_id>10011007.10011006.10011008</concept_id>
%<concept_desc>Software and its engineering~General programming languages</concept_desc>
%<concept_significance>500</concept_significance>
%</concept>
%<concept>
%<concept_id>10003456.10003457.10003521.10003525</concept_id>
%<concept_desc>Social and professional topics~History of programming languages</concept_desc>
%<concept_significance>300</concept_significance>
%</concept>
%</ccs2012>
\end{CCSXML}

%\ccsdesc[500]{Software and its engineering~General programming languages}
%\ccsdesc[300]{Social and professional topics~History of programming languages}
%% End of generated code
\maketitle




\section{Introduction}

The great proliferation of computer music programming languages points to the difficulty of building a natural interface for users that want to computationally interact with musical data.
Programming applications in the domain of computer music, and specifically digital signal processing (DSP), requires that users not only grasp fundamental programming techniques, but also have a large domain specific knowledge on time and signal manipulations.
The amount of prerequisite skill and effort to overcome these barriers is often higher than many users are able to commit.

Furthermore, the difficult of programming DSP applications is often not commensurate with the creative intentions.
From a musical perspective, take the following simple use-case: a user hears a sample in a piece of music, and later in that piece hears the same sample with some added effects.
Now the user wishes to apply this same effect to their own sample.
In order to recreate this effect on a new sample, the user will have to reprogram the filter from the scratch.
This process will involve writing code, testing the code, listening to the original example, and constantly tweaking parameter values.

To simplify this process, we introduce \textit{DSP programming by example} (DSP-PBE).
In DSP programming by example, a user directly provides an input sound sample and an output sound sample, and a DSP-PBE tool will automatically construct the required program that represents a DSP filter that can transform the input sound into the output sound.

\subsection{Program Synthesis}
Programming by example, and program synthesis more generally, has experienced an explosion of research and progress in the last 15 years within the formal logic research community.
This has led to real world applications, such as FlashFill, a programming by example plugin for Microsoft Excel~\cite{flashfill}.


\section{Motivating Example}

As a motivating example, imagine a user was to reconstruct the filter that was used to transform an audio clip, as shown in Figure~\ref{fig:test}.
In this example, a user provided a clip of a \texttt{cartoon-spring.wav} in Figure~\ref{fig:inEx}, and the same sound as it had been transformed with a low-pass filter at 800 Hz, $lpf(800)$, as shown in Figure~\ref{fig:outEx}.
However the nature of the transformation is unknown to the user and they wish to discover the filter needed.
Our DSP-PBE tool is able to synthesize a filter $lpf(947)$, that when applied to the original sound, produces the waveform shown in Figure~\ref{fig:synthEx}.
While the solution is not exact, the difference is not significantly noticeable to an untrained ear.

\begin{figure}
\centering
\begin{subfigure}{.32\linewidth}
  \centering
  \includegraphics[width=.9\textwidth]{figs/original.png}
  \caption{Input example}
  \label{fig:inEx}
\end{subfigure}%
\begin{subfigure}{.32\linewidth}
  \centering
  \includegraphics[width=.9\textwidth]{figs/lpf800.png}
  \caption{Output example}
  \label{fig:outEx}
\end{subfigure}
\begin{subfigure}{.32\linewidth}
  \centering
  \includegraphics[width=.9\textwidth]{figs/lpf950.png}
  \caption{Generated}
  \label{fig:synthEx}
\end{subfigure}
\caption{The waveforms (a) and (b) are provided as examples, and DSP-PBE synthesizes a filter that produces (c).}
\label{fig:test}
\end{figure}



\section{Background}
\label{sec:background}

To give context for DSP-PBE, we first explain the traditional concept of programming by example~\cite{cypher93,lieberman01,synasc12}.
Programming by example (PBE) is a synthesis technique that automatically generates programs that coincide with given examples.
An example is specified as a tuple of input and output values.
Given a set $S= \{(i_1, o_1),\ldots, (i_n, o_n)\}$ of input/output examples, the goal is to automatically derive a program $P$ such that for every $j$, $P(i_j) = o_j$.

PBE is in line with one of the often repeated high level goals of functional programming -- to describe \textit{what} a program should do, and not \textit{how} the program should do it.
Instead of writing code, the user provides a list of relevant examples and the synthesis tool automatically generates a program.
In this way, the examples can be seen as an easily readable and understandable specification.
However, even if the synthesized program satisfies all the provided examples, it still might not correspond to the user's intentions.
Examples are, by nature, an incomplete specification.

PBE is a promising research direction that enables easy manipulation of data even for non-programmers~\cite{GulwaniHS12}.
Recent work in this area has focused on manipulating fundamental data types such as strings~\cite{vldb12,icml13} and lists~\cite{FeserCD15,poseraZ15}.
The success and impact of this line of work can be estimated from the fact that PBE ships as part of the popular Flash Fill feature in Excel 2013~\cite{flashfill}.


The core difference between traditional PBE and DSP-PBE is in the application domain of Digital Signal Processing.
Digital Signal Processing (DSP) programming languages provide users with an interface to build signal processing programs in domain specific languages.
Some of these languages provide their own implementations of signal processing primatives, such as SuperCollider~\cite{supercollider}, CSound~\cite{csound}, and PureData~\cite{puredata}.
Other DSP languages provide alternative front-ends to these languages, such as Vivid~\cite{vivid}, which provides Haskell bindings to Supercollider.

Although many DSP languages are full featured enough to write general purpose programs, in this work we focus on the construction of DSP filters.
A DSP filter is, broadly speaking, any program that transforms a digital signal from one form to another.
An example of a DSP filter is a low-pass filter, which takes an input signal and generates an output signal that keeps frequencies below some frequency threshold, but removes frequencies above that threshold.

The most closely related work in audio signal processing is a technique called resynthesis~\cite{masri1996improved}.
Resynthesis is the process of decomposing a sound into its spectrogram, and then building a synthesizer to recreate a similar sound.
The limitation here is that resynthesis builds a generative synthesizer, which does not take into account any information about the components used to create the original sound.
This limitation means that resynthesis cannot be applied in a new context, whereas DSP-PBE allows us to construct a DSP program that can be used with various new input samples to create novel sounds. 
For example, DSP-PBE could be given a sample of a trumpet and a trombone, and the generated DSP program could be applied to a violin to hear what a violin sounds like if it was a trumpet that had been turned into a trombone.
In this case we can discover the analogy \textit{trumpet:trombone :: violin:?}.

From a machine learning perspective, the above example use case is closely related to work on learning analogies~\cite{mikolov2013distributed}, where the goal is to discover relations such as \textit{man:king :: woman:queen}.
To do this, words are embedded in a vector space, so that the transformation from \textit{man} to \textit{king}, can be directly applied to \textit{woman}.
There are two keys differences between this approach and DSP-PBE.
The first is that DSP-PBE should produce a human readable transformation.
We would like to generate DSP programs that can be used verbatim, but also inspected and modified by the user.
While program code provides this readability, vector transformations are not comprehensible in the same way. 
Second, word embeddings require that the semantics of an object can be embedded into a vector space. 
As we will see in Sec~\ref{sec:distance}, a semantic representation of an audio file (what a human perceives) is not immediately recoverable from its direct representation.


\section{Aural Distance}

As a distance metric, we used as a starting point the literature on acoustic fingerprinting.
Acoustic fingerprinting is the concept of creating a condensed, distinct summary of an audio file that can be used later to identify that audio file or to look it up in a database.
Acoustic fingerprints represent how the file will sound to the human ear regardless of how it is represented in a digital format.
There are numerous ways to develop acoustic fingerprints and companies like Shazam and Sound-Hound have developed complex algorithms to create accurate fingerprints even from low quality files recorded on a cellphone mic.
One of Shazam’s engineers released a paper detailing their process several years ago.
I used this paper as an inspiration for my checker.
The Shazam method uses Fast Fourier Transforms to convert the WAVE file into a spectrogram plotted over time.
Then they plot the (frequency,time) coordinates with the greatest amplitude to create a star map of points that become the files audio fingerprint.
I use a similar strategy by first performing a real Fast Fourier Transform on the imported WAVE file and then picking out the frequency peaks in each time frame.


Fast Fourier Transforms are the key to a good acoustic fingerprint.
Prior to this project, I had only heard of Fourier Transforms in passing and had no idea what they did or how they worked.
I was able to find several Haskell libraries that provide DFT and FFT functions but the majority of my reading on this topic was on how to interpret the results of these functions.
The results of these functions are not presented in an intuitive manner and one must take into account negative frequencies.
The return arrays are presented with the zero frequency component or DC, followed by positive frequencies up until the Nyquist frequency (sampling frequency divided by 2) after which all the elements represent negative frequencies which are irrelevant to a spectrogram.
For this reason, I had to adjust the checker to only observe the array elements from to the Nyquist frequency and exclude the zeroth element and all the negative frequency.

In my readings about the results of FFT I came across the topic of spectral leakage.
Frequency and time are continuous spectra.
FFT, however, relies on a discrete representation of the sound wave over time and therefore returns a discrete representation of the frequency spectrum.
Each return element is a frequency ”bin”, and depending on the scale of your return array the size of this bins varies.
In order for each bin to correspond to 1 Hz the size of the return vector must be equal to the sampling frequency (44,100 Hz).
If each bin is not 1 Hz, you can expect to see the effects of spectral leakage.
This occurs when the bins do not correspond to the exact frequency peaks of the sound.
The amplitude from the peaks that fall in between bins will ”leak” over into the closest bin and create a distorted spectrogram.
For this reason I had to adjust the size of the FFT return arrays to be 44,100.
Although this slows down the process of FFT, it provides the most accurate representation of the sound and for our purposes frequency accuracy is paramount.



\section{Search}
\label{sec:search}

As the search space of possible DSP program is extremely large, our search procedures must be exceptionally efficient. 
As a first foray into DSP-PBE, we restrict ourselves to only synthesizing low-pass and high-pass filters, and global volume adjustment.
These two filters have the key property that they are quasi-commutative - when the thresholds of these filters do not overlap, applying a low-pass and then a high-pass is the same as applying a high-pass and then a low-pass.
We leave the exploration of non-commutative filters (for example, delay lines or ring filters) to future work.

\subsection{Gradient Descent}

Gradient descent is a technique commonly used in modelling and machine learning.
Given a cost function, which represents the disagreement between a proposed model and the actual data, gradient descent can be used efficiently to minimize the cost and generate the model of best fit.
One key restriction on the cost function is that it must be convex - this is because gradient descent will ``descend'' along the surface of the cost function, in each step following the steepest gradient.
We have carefully designed our aural distance function from Section~\ref{sec:distance} to be as close to convex as possible.

\begin{figure}[!h]
\includegraphics[width=\columnwidth]{figs/distCurves} 
\caption{The distance curves showing the convex-like shape of the aural distance function. Each curve is the $dist(\synthFilter(I),O)$, with $\synthFilter = lpf(x)$ where the value of $x$ is taken from the x-axis.}
\label{fig:distCurves}
\end{figure}

In order to visualize the convexity of our distance metric, we plot the distance between pairs of examples, and various possible DSP filters in Figure~\ref{fig:distCurves}.
Here we only visualize the distance curves in the dimension of the low-pass filter.
Notice that the curves exhibit a clear ``saddle'', which represents the minimum cost.
In the ideal case, gradient descent will find these points.
Note that we do not have these graphs available during synthesis - producing the entire graph as in Figure~\ref{fig:distCurves} is prohibitively expensive.

In Figure~\ref{fig:distCurves}, the last curve we plot is the distance between \texttt{cartoon-spring.wav} and \texttt{cartoon-spring-hpf1500.wav}, the same file with a high pass filter applied with a threshold of 1500 Hz.
Notice that as the threshold of the low-pass filter applied to the input example (\texttt{cartoon-spring.wav} increases, the distance to the output example decreases. 
This is because as a low-pass filter's threshold increases, it allows more and more frequencies to pass into the output - thereby having less of an effect.
Whereas in the case of the \texttt{cartoon-spring-hpf1500.wav}, the true filter is a high-pass filter, so the less we apply a low-pass filter, the closer we get to the correct filter.

\begin{figure}[!h]
\includegraphics[width=\columnwidth]{figs/distCurveZoom} 
\caption{Zooming in (1000 to 1500 Hz) on a portion of a curve from Figure~\ref{fig:distCurves}, we see the aural distance function is not perfectly convex on the micro scale.}
\label{fig:microDist}
\end{figure}

There are a number challenges with working with gradient descent in the aural DSP domain.
The first is that the domain can have a high number of dimensions to search.
In our implementation, we only explore a two DSP filters and volume adjustment, but this results in 5 dimensional space (each filter requires both a threshold value and an amplitude value for how much of the filter to apply).
To speed up gradient descent, we use stochastic gradient descent, so that in each step, we only move in $d<5$ number of dimensions.

Additionally, on the micro scale, the distance function is susceptible to noise and not entirely smooth, as shown in Figure~\ref{fig:microDist}.
In order to handle the micro scale variations, we use a periodic restart of the gradient descent.
This means that every $n$ rounds, as defined by the user (we use 4), the gradient descent will backtrack to the best solution it has found so far.
The stochastic gradient descent will then continue, selecting dimensions to explore in each round using a new random seed.

\begin{figure}[!h]
\includegraphics[width=\columnwidth]{figs/distCurveMacro} 
\caption{Looking at the portion of a curve from Figure~\ref{fig:distCurves} between 8k Hz and 20k Hz, we see the aural distance function is not perfectly convex on the macro scale. In this case, that is because the sample has very few frequencies above the 8k Hz range.}
\label{fig:macroDist}
\end{figure}

On the macro scale, we face the challenge that the distance function is only quasi-convex - there are many local minimum and long plateaus, as shown in Figure~\ref{fig:macroDist}.
In order to overcome this, we must carefully pick the initial value for gradient descent.
If we pick a value in the middle of a plateau, the gradient descent algorithm will not find any significant gradient, and conclude we have reached the convergence condition.
In our current implementation, we iterate at large intervals (1000 Hz) possible threshold values for both low and high pass filters.
We choose possible DSP programs that use only low pass, only high pass, and both low and high pass filters.
After evaluating these, we take the lowest cost initial DSP program, and start gradient descent from that point.

Finally, one of the key parts of a good application of gradient descent is the choice of the parameters such as the learning rate and the convergence goal.
These parameters must be adjusted based on the values observed from the cost (in our case, distance) function.
While the details of tuning gradient descent are outside the scope of this paper, it suffices to note that any change in the distance metric will likely also require a readjustment of these parameters.



\section{Evaluation}
\label{sec:eval}
\begin{table*}[!ht]
\centering
\setlength\tabcolsep{2em}
\begin{tabular}{|c | c | c | c |} 
 \hline
 Description & True DSP & Synth'ed DSP & Time (sec) \\ 
 \hline\hline
 Cartoon Spring & $lpf(800)  $ & $lpf(1989) $ & 56.195 \\
 Cartoon Spring & $lpf(5000) $ & $lpf(4000) \arrComp hpf(7000) $ & 54.004 \\
 Cartoon Spring & $hpf(1500) $ & $lpf(1000) \arrComp hpf(1000) $ & 53.964 \\
 BTS DNA (Kpop) & $lpf(2000) $ & $lpf(1996)	 $ & 56.874 \\
 Holst Mars     & $hpf(3500) $ & $lpf(10000) \arrComp hpf(1000)$  & 55.444 \\
 \hline
\end{tabular}
\caption{Time to converged on a solution DSP program for various benchmarks. The programs may not match the known DSP program, but may still be psycho-acoustically equivalent depending on the expertise of the listener. }
\label{table:eval}
\end{table*}

We implemented a DSP-PBE tool based on the approach described in Section~\ref{sec:distance} and Section~\ref{sec:search}.
Our tool is available open-source at~\url{www.github.com/santolucito/DSP-PBE}.
Our tool is mostly written in Haskell and uses the Vivid library~\cite{vivid} for bindings to SuperCollider~\cite{supercollider}.
Haskell allows easy access to type information and metaprogram construction tools that are useful for program synthesis, however the programs themselves are easily translated back to SuperCollider ``synth defs'', which are DSP filter programs.
We use the scipy python module for calling the FFT since the library is quite mature and provides a simplified interface specifically for calling FFT on audio.

One key implementation point is that we use a separate representation of a DSP for running gradient descent, and for actually processing the audio.
Gradient descent works best when all parameters are in the same scale, so we map the frequencies [0,20k] Hz to a [-1,1] scale.
Likewise, we map the application levels for each filter (how much of the filtered output should be included in the final mix) on a [-1,1] scale.

In Table~\ref{table:eval}, we show the results of running our tool on a set of benchmarks of input/output example audio samples.
The audio samples were transformed in Audacity, using the Low Pass Filter and High Pass Filter effects.
Since we use SuperCollider's filter implementations on the backend, there may be very slight variation, but this is to be expected in real-world application as well.
All experiments were run on an Intel Core i7-6820HQ CPU @ 2.70GHz with 16 GB of RAM and an Intel Sunrise Point-H HD Audio sound card.


We can also breakdown the runtime cost of synthesis into the two different stages - 1) initial program selection, and 2) gradient descent.
The initial program selection phase is a mostly fixed cost, as we always evaluate the same distribution of initial value.
On average this process takes roughly 40 seconds.
We outline future directions of research that may be able to reduce this cost in Section~\ref{sec:rtypes}.


\section{Refinement Type Driven Synthesis}
\label{sec:rtypes}

In order to find an initial value for gradient descent, we could use refinement types.
In this section we explore a possible optimization for selecting an initial DSP program for gradient descent.
This has not yet been implemented, but we present the theory behind the approach.

\subsection{Refinement Types for DSP}
\label{sec:retypeSearch}
Refinement types are a way of giving an abstract description of the behavior of a function. 
For example, given the function 
%
\texttt{map :: [a]} $\to$ \texttt{[b]}
%
we can further provide a refinement types that captures some properties of the behavior of this function over values:
%
\begin{align*}
\texttt{f :: xs:[a]} \to\ \texttt{ys:[b]}\ \mid \texttt{ length xs == length ys}
\end{align*}

\noindent In this case, the refinement type describes that the length of the lists are still equal after applying the \texttt{map} function.

In a similar style for DSP, we can write predicates about the filters available to us during synthesis. 
For example, a low-pass filter could be described as the refinement type that says the amplitude of the frequencies greater than the threshold frequency have decreased in the output Audio.
For brevity in notation, we will only treat a single time slice from the waterfall plot here, but the concept generalizes when quantified over all time slices as well.
%
\begin{align*}
  &\texttt{lpf :: t:Float} \to  \texttt{xs:Audio} \to\ \texttt{ys:Audio}\ \mid \\
  &\forall f_1 \in  \texttt{spectrogram(xs)}.\ \forall f_2 \in \texttt{spectrogram(ys)}. \\
  &(f_1 > \texttt{t}  \land  f_2 > \texttt{t}  \land f_1 == f_2) \implies \texttt{amp}(f_1) > \texttt{amp}(f_2)
\end{align*}

Where \texttt{t} represents the level at which the lowpass filter is applied, \texttt{spectrogram} represents the spectrogram of the sound sample, $f_i$ represents a frequency, and \texttt{amp()} represents the amplitude of the frequency. 

Additionally, a high-pass filter could be described as the refinement type that says the amplitude of the frequencies less than the threshold frequency have decreased in the output Audio.
%
\begin{align*}
  &\texttt{hpf :: t:Float} \to\ \texttt{xs:Audio} \to\ \texttt{ys:Audio}\ \mid \\
  &(\forall f_1 \in \texttt{spectrogram(xs)}. \forall f_2 \in \texttt{spectrogram(ys)}). \\
  &(f_1 < \texttt{t} \land f_2 < \texttt{t} \land f_1 == f_2) \implies \texttt{amp(}f_1\texttt{)} > \texttt{amp(} f_2 \texttt{)} 
\end{align*}

Notice that in these refinement types, we only need to calculate the spectrogram for the input and output statically.
As opposed to the current technique of generating filters, applying them, and the calculating the aural distance, this approach is relatively static.
We could quickly check many threshold values over the input and output examples.
This will only yield a rough boolean estimation of whether this threshold should even be considered, but this is enough information for us to select an initial program to pass to our gradient descent algorithm.
As the search for an initial filter takes roughly 40 seconds out of our current benchmarks, this could dramatical increase the speed of synthesis.

\subsection{Combination of Search Algorithms}

Beyond just using the refinement types to select an initial program for gradient descent, we can use refinement types in as part of the main search strategy as well.
We briefly describe here a way to use refinement types in combination with gradient descent to handle more complex combinations of DSP filters.
So far in our work (\textit{c.f.} Sec.~\ref{sec:eval}) we have synthesized filters with a fixed form - all our solutions use a single low-pass filter, and a single high-pass filter.
Ideally, we would be able to synthesize solutions that use any arbitrary combination of filters.
In order to do this, we would need an iterative solution that can find one filter at a time.

In this approach, given input example \texttt{x:Audio} and output example \texttt{y:Audio}, we would first find a filter \synthFilter' using the approach described in Sec.~\ref{sec:search} and Sec.~\ref{sec:rtypeSearch}.
We will say that this \synthFilter' has the refinement type $r_1$.
However, this filter might not return a satisfactory result.
We could then continue the search using the output of \synthFilter'\texttt{(x)} as the new input example, \texttt{z:Audio}.
Now the synthesis task is to find a filter \synthFilter'' (with refinement type $r_2$) using input example \texttt{z:Audio} and output example \texttt{y:Audio}.
Essentially, \synthFilter' has gotten us the first half of the way, and \synthFilter'' will get us the second half of the way.
With this, we can start to use more information rich refinement types, such as below:
%
\begin{align*}
\synthFilter \texttt{:: } & \texttt{x:Audio} \to\ \texttt{y:Audio}\ \mid \\
   & \exists \ \texttt{z:Audio}.\ r_1\texttt{(x,z)} \land r_2\texttt{(z,y)}
\end{align*}


\section{Future Work and Conclusions}

The main contribution of this paper is to pose the problem of DSP-PBE.
While we have presented a prototype implementation of a DSP-PBE tool, this primarily functions as a proof-of-concept.
There remains significant room for optimization in both the distance calculation and the search algorithm.

Although with the current tool, synthesis times presented might be prohibitively slow for many use cases, we should be encouraged by progress in other domains of program synthesis.
In the SyGuS program synthesis competition, which has run for 4 years, tools have seen an exponential speed up and increase in the range of programs that can be synthesized.
As one example, in the 2014 competition the \texttt{LinExpr\_eq1.sl} benchmark was only solved by one tool, and took 1128 seconds in 2014~\cite{sygus2014}.
In the 2017 competition, the same benchmark was solved by all tools, with the fastest taking only 199 seconds~\cite{sygus2017}.




\bibliography{nime-references}  

%%% Place this command where you want to balance the columns on the last page. 
%\balancecolumns 

% That's all folks!
\end{document}
