\section{Future Work and Conclusions}

The main contribution of this paper is to pose the problem of DSP-PBE.
While we have presented a prototype implementation of a DSP-PBE tool, this primarily functions as a proof-of-concept.
There remains significant room for optimization in both the distance calculation and the search algorithm.
In future work, we also plan to expand beyond commutative filters to be able to synthesize effects such as delay lines.

We briefly revisit the motivating conceptual example from Sec.~\ref{sec:background} where we want to synthesize the filter that transforms a trumpet into a trombone and apply that filter to a violin.
In order to achieve this we need more complex filters than high-pass, low-pass, and amplitude adjustment.
Despite a lack of a formal approach for non-commutative filters, we added a pitch shift filter and the \texttt{ringz} delay line filter from SuperCollider into our tool using the current approach.
Though we had no reason to believe that our current approach should produce useful results when allowing more complex filters in the search space of synthesis, we found the results to be at least interesting, and even reasonable.
The synthesized filter to transform a trumpet into a trombone was as follows:

\begin{lstlisting}
SetVolume: 1.00% 
LoPass: freq@1000 amp@0.91 >>> 
HiPass: freq@100 amp@0.00 >>> 
PitchShift: freq@-1600 amp@0.91 >>> 
Ringz: freq@9100 delay@0.10 amp@0.23 >>> 
WhiteNoise: amp@0.00
\end{lstlisting}

The trumpet and trombone input audio files, as well as the audio of a violin with this filter applied is available as a Soundcloud playlist~\cite{soundcloudAudio}.

Although with the current tool, synthesis times presented might be prohibitively slow for many use cases, especially on such small programs, we should be encouraged by progress in other domains of program synthesis.
In the SyGuS program synthesis competition, which has run for four years, tools have seen an exponential speed up and increase in the range of programs that can be synthesized.
As one example, in the 2014 competition the \texttt{LinExpr\_eq1.sl} benchmark was only solved by one tool, and took 1128 seconds~\cite{sygus2014}.
In the 2017 competition, the same benchmark was solved by all tools, with the fastest taking only 199 seconds~\cite{sygus2017}.
