% ===========================================
% Template for ICMC-NYCEMF 2019 (version2)
% adapted from earlier LaTeX paper templates for the ICMC, SMC, etc...
% ===========================================

\documentclass{article}
\usepackage{icmc2019template}
\usepackage{times}
\usepackage{ifpdf}
\usepackage{soul}
\usepackage[english]{babel}
%\usepackage{cite}



\usepackage{xspace}
\usepackage{caption}
\usepackage{subcaption}


\newcommand{\synthFilter}{\ensuremath{\mathcal{F}}\xspace}


%%%%%%%%%%%%%%%%%%%%%%%% Some useful packages %%%%%%%%%%%%%%%%%%%%%%%%%%%%%%%
%%%%%%%%%%%%%%%%%%%%%%%% See related documentation %%%%%%%%%%%%%%%%%%%%%%%%%%
%\usepackage{amsmath} % popular packages from Am. Math. Soc. Please use the 
%\usepackage{amssymb} % related math environments (split, subequation, cases,
%\usepackage{amsfonts}% multline, etc.)
%\usepackage{bm}      % Bold Math package, defines the command \bf{}
%\usepackage{paralist}% extended list environments
%%subfig.sty is the modern replacement for subfigure.sty. However, subfig.sty 
%%requires and automatically loads caption.sty which overrides class handling 
%%of captions. To prevent this problem, preload caption.sty with caption=false 
%\usepackage[caption=false]{caption}
%\usepackage[font=footnotesize]{subfig}

% ====================================================
% ================ Define title and author names here ===============
% ====================================================
%user defined variables
\def\papertitle{Synthesizing DSP Filters from Examples of Analog Effects}
\def\firstauthor{First Author}
\def\secondauthor{Second Author}
\def\thirdauthor{Third Author}
\def\fourthauthor{Fourth Author}
\def\fifthauthor{Fifth Author}
\def\sixthauthor{Sixth Author}

% adds the automatic
% Saves a lot of output space in PDF... after conversion with the distiller
% Delete if you cannot get PS fonts working on your system.

% pdf-tex settings: detect automatically if run by latex or pdflatex
\newif\ifpdf
\ifx\pdfoutput\relax
\else
   \ifcase\pdfoutput
      \pdffalse
   \else
      \pdftrue
  \fi
\fi

\ifpdf % compiling with pdflatex
  \usepackage[pdftex,
    pdftitle={\papertitle},
    pdfauthor={\firstauthor, \secondauthor, \thirdauthor},
    bookmarksnumbered, % use section numbers with bookmarks
    pdfstartview=XYZ % start with zoom=100% instead of full screen; 
                     % especially useful if working with a big screen :-)
   ]{hyperref}
  %\pdfcompresslevel=9

  \usepackage[pdftex]{graphicx}
  % declare the path(s) where your graphic files are and their extensions so 
  %you won't have to specify these with every instance of \includegraphics
  \graphicspath{{./figures/}}
  \DeclareGraphicsExtensions{.pdf,.jpeg,.png}

  \usepackage[figure,table]{hypcap}

\else % compiling with latex
  \usepackage[dvips,
    bookmarksnumbered, % use section numbers with bookmarks
    pdfstartview=XYZ % start with zoom=100% instead of full screen
  ]{hyperref}  % hyperrefs are active in the pdf file after conversion

  \usepackage[dvips]{epsfig,graphicx}
  % declare the path(s) where your graphic files are and their extensions so 
  %you won't have to specify these with every instance of \includegraphics
  \graphicspath{{./figures/}}
  \DeclareGraphicsExtensions{.eps}

  \usepackage[figure,table]{hypcap}
\fi

%setup the hyperref package - make the links black without a surrounding frame
\hypersetup{
    colorlinks,%
    citecolor=black,%
    filecolor=black,%
    linkcolor=black,%
    urlcolor=black
}


% ====================================================
% ================ Title and author info starts here ===============
% ====================================================
% Title.
% ------
\title{\papertitle}

% Authors
% Please note that submissions are anonymous, therefore 
% authors' names should not be VISIBLE in your paper submission.
% They should only be included in the camera-ready version of accepted papers.
% uncomment and use the appropriate section (1, 2 or 3 authors)
%
% Single address
% To use with only one author or several with the same address
% ---------------
%\oneauthor
%   {\firstauthor} {Affiliation \\ %
%     {\tt \href{mailto:author@unt.edu}{author@unt.edu}}}

%Two addresses
% the default spacing is 1.5in, but this can be reduced to 0.5in or less, if needed
%--------------
% \twoauthors
%   {1.5in}
%   {\firstauthor} {Affiliation1 \\  %
%     {\tt \href{mailto:author1@unt.edu}{author1@unt.edu}}}
%   {\secondauthor} {Affiliation2 \\  %
%     {\tt \href{mailto:author2@unt.edu}{author2@unt.edu}}}

% Three addresses
% the default spacing is 0.5in, but this can be reduced to 0.3in or less, if needed
% --------------
 \threeauthors
   {0.5in}
   {\firstauthor} {Affiliation1 \\ %
     {\tt \href{mailto:author1@smcnetwork.org}{author1@smcnetwork.org}}}
   {\secondauthor} {Affiliation2 \\ %
     {\tt \href{mailto:author2@smcnetwork.org}{author2@smcnetwork.org}}}
   {\thirdauthor} { Affiliation3 \\ %
     {\tt \href{mailto:author3@smcnetwork.org}{author3@smcnetwork.org}}}

% Four addresses
% the default spacing is 1.5in, but this can be reduced to 0.5in or less, if needed
% --------------
% \fourauthors
%   {1.5in}
%   {\firstauthor} {Affiliation1 \\ %
%     {\tt \href{mailto:author1@unt.edu}{author1@unt.edu}}}
%   {\secondauthor} {Affiliation2 \\ %
%     {\tt \href{mailto:author2@unt.edu}{author2@unt.edu}}}
%   {\thirdauthor} { Affiliation3 \\ %
%     {\tt \href{mailto:author3@unt.edu}{author3@unt.edu}}}
%   {\fourthauthor} { Affiliation4 \\ %
%     {\tt \href{mailto:author4@unt.edu}{author4@unt.edu}}}

% Five addresses
% the default spacing is 0.5in, but this can be reduced to 0.3in or less, if needed
% --------------
% \fiveauthors
%   {0.5in}
%   {\firstauthor} {Affiliation1 \\ %
%     {\tt \href{mailto:author1@unt.edu}{author1@unt.edu}}}
%   {\secondauthor} {Affiliation2 \\ %
%     {\tt \href{mailto:author2@unt.edu}{author2@unt.edu}}}
%   {\thirdauthor} { Affiliation3 \\ %
%     {\tt \href{mailto:author3@unt.edu}{author3@unt.edu}}}
%   {\fourthauthor} { Affiliation4 \\ %
%     {\tt \href{mailto:author4@unt.edu}{author4@unt.edu}}}
%   {\fifthauthor} { Affiliation5 \\ %
%     {\tt \href{mailto:author5@unt.edu}{author5@unt.edu}}}

% Six addresses
% the default spacing is 0.5in, but this can be reduced to 0.3in or less, if needed
% --------------
% \sixauthors
%   {0.5in}
%   {\firstauthor} {Affiliation1 \\ %
%     {\tt \href{mailto:author1@unt.edu}{author1@unt.edu}}}
%   {\secondauthor} {Affiliation2 \\ %
%     {\tt \href{mailto:author2@unt.edu}{author2@unt.edu}}}
%   {\thirdauthor} { Affiliation3 \\ %
%     {\tt \href{mailto:author3@unt.edu}{author3@unt.edu}}}
%   {\fourthauthor} { Affiliation4 \\ %
%     {\tt \href{mailto:author4@unt.edu}{author4@unt.edu}}}
%   {\fifthauthor} { Affiliation5 \\ %
%     {\tt \href{mailto:author5@unt.edu}{author5@unt.edu}}}
%   {\sixthauthor} { Affiliation6 \\ %
%     {\tt \href{mailto:author6@unt.edu}{author6@unt.edu}}}


% ====================================================
% =============== The document content starts here ===============
% ====================================================
\begin{document}
%
\capstartfalse
\maketitle
\capstarttrue
%
\begin{abstract}
Many computer music languages provide a way to write digital signal processing filters, which can be used to model analog effects such as a trumpet mute, or a voice underwater.
Writing filters that correspond to these effects is a difficult task that requires an deep musical understanding of digital signal processing for audio and domain expertise in the audio programming language of choice.
In order to overcome this challenge, we present a tool, SynthSynth, that allows beginners to construct DSP filters simply by providing example audio files.
We present some preliminary results of applying SynthSynth to the reconstruction of analog effects.
We further explain how we generate code for languages such as SuperCollider and MaxMSP from our internal representation of a DSP filter.
\end{abstract}
%

\section{Introduction}

The great proliferation of computer music programming languages points to the difficulty of building a natural interface for users that want to computationally interact with musical data.
Programming applications in the domain of computer music, and specifically digital signal processing (DSP), requires that users not only grasp fundamental programming techniques, but also have a large domain specific knowledge on time and signal manipulations.
The amount of prerequisite skill and effort to overcome these barriers is often higher than many users are able to commit.

Furthermore, the difficult of programming DSP applications is often not commensurate with the creative intentions.
From a musical perspective, take the following simple use-case: a user hears a sample in a piece of music, and later in that piece hears the same sample with some added effects.
Now the user wishes to apply this same effect to their own sample.
In order to recreate this effect on a new sample, the user will have to reprogram the filter from the scratch.
This process will involve writing code, testing the code, listening to the original example, and constantly tweaking parameter values.

To simplify this process, we introduce \textit{DSP programming by example} (DSP-PBE).
In DSP programming by example, a user directly provides an input sound sample and an output sound sample, and a DSP-PBE tool will automatically construct the required program that represents a DSP filter that can transform the input sound into the output sound.

\subsection{Program Synthesis}
Programming by example, and program synthesis more generally, has experienced an explosion of research and progress in the last 15 years within the formal logic research community.
This has led to real world applications, such as FlashFill, a programming by example plugin for Microsoft Excel~\cite{flashfill}.





\section{Related}

Previously, the machine learning and semantic modelling approach to program synthesis have been viewed as at odds with one another.
Depending on the context, one or the other outperforms the other~\cite{devlin2017robustfill}.
In this work, we join the two approaches together in a restricted example domain (where the example domain forms a metric space).

The work of~\cite{misailovic2011probabilistically} introduces a notion of approximate correctness of program transformation over probabilistic programs.
Aside from our work focusing on synthesis, we also are working over deterministic programs - our notion of approximate is not with respect to probabilistic outcomes, but closeness in the metric space.



\section{System Overview}

How much of this should be the algorithm and how much should be implementation specific

\markk{system image here specialized on analog effects}

\subsection{Aural Distance as a Metric Space}

One of the key components in machine learning systems is having a metric that to quantifies how close the candidate solution is to the desired solution.
In the case of programming-by-example, the desired solution is defined by the user-provided input-output example pairs.
At a high-level, our goal is to minimize the distance, $dist$, between the output of the candidate function, $F$, on the provided input, $I$, and the provided output $O$.
Put mathematically, we have a minimization problem: $min (dist ( F(I)), O)$.

Previous work proposed using a distance metric between two audio files based on constellation plots of multiple FFT plots over the samples~\cite{SantolucitoFARM}.
We adopt the same distance metric for this work.

\subsection{Gradient Descent}

We use a modified version of gradient descent to find good parameters for the filter.
Have to decide how much to go into this - probably too technical for ICMC, but there are few things that are new from FARM that would be nice to highlight

\subsection{Choosing an initial starting point}

As a very rough estimate, if the max freq peak of the output is less than the max freq peak of input
  we need a lpf, and it should have a value a bit less than the max peak of output

\subsection{Searching for DSP structure}

Just brute force for now - maybe guided by user? 
Something smarter can come later.

\subsection{Generating Program Code}

\markk{how to we turn the solution into a PD/MaxMSP/SuperCollider program (see TODO on github, still need to actually code this)}


\section{Evaluation}
\label{sec:eval}
\begin{table*}[!ht]
\centering
\setlength\tabcolsep{2em}
\begin{tabular}{|c | c | c | c |} 
 \hline
 Description & True DSP & Synth'ed DSP & Time (sec) \\ 
 \hline\hline
 Cartoon Spring & $lpf(800)  $ & $lpf(1989) $ & 56.195 \\
 Cartoon Spring & $lpf(5000) $ & $lpf(4000) \arrComp hpf(7000) $ & 54.004 \\
 Cartoon Spring & $hpf(1500) $ & $lpf(1000) \arrComp hpf(1000) $ & 53.964 \\
 BTS DNA (Kpop) & $lpf(2000) $ & $lpf(1996)	 $ & 56.874 \\
 Holst Mars     & $hpf(3500) $ & $lpf(10000) \arrComp hpf(1000)$  & 55.444 \\
 \hline
\end{tabular}
\caption{Time to converged on a solution DSP program for various benchmarks. The programs may not match the known DSP program, but may still be psycho-acoustically equivalent depending on the expertise of the listener. }
\label{table:eval}
\end{table*}

We implemented a DSP-PBE tool based on the approach described in Section~\ref{sec:distance} and Section~\ref{sec:search}.
Our tool is available open-source at~\url{www.github.com/santolucito/DSP-PBE}.
Our tool is mostly written in Haskell and uses the Vivid library~\cite{vivid} for bindings to SuperCollider~\cite{supercollider}.
Haskell allows easy access to type information and metaprogram construction tools that are useful for program synthesis, however the programs themselves are easily translated back to SuperCollider ``synth defs'', which are DSP filter programs.
We use the scipy python module for calling the FFT since the library is quite mature and provides a simplified interface specifically for calling FFT on audio.

One key implementation point is that we use a separate representation of a DSP for running gradient descent, and for actually processing the audio.
Gradient descent works best when all parameters are in the same scale, so we map the frequencies [0,20k] Hz to a [-1,1] scale.
Likewise, we map the application levels for each filter (how much of the filtered output should be included in the final mix) on a [-1,1] scale.

In Table~\ref{table:eval}, we show the results of running our tool on a set of benchmarks of input/output example audio samples.
The audio samples were transformed in Audacity, using the Low Pass Filter and High Pass Filter effects.
Since we use SuperCollider's filter implementations on the backend, there may be very slight variation, but this is to be expected in real-world application as well.
All experiments were run on an Intel Core i7-6820HQ CPU @ 2.70GHz with 16 GB of RAM and an Intel Sunrise Point-H HD Audio sound card.


We can also breakdown the runtime cost of synthesis into the two different stages - 1) initial program selection, and 2) gradient descent.
The initial program selection phase is a mostly fixed cost, as we always evaluate the same distribution of initial value.
On average this process takes roughly 40 seconds.
We outline future directions of research that may be able to reduce this cost in Section~\ref{sec:rtypes}.


\section{Conclusions}

With \ourTool, we generate SuperCollider code from input and output example files, assisting novice DSP programmers in creating audio effects that are unfamiliar to them.
We produce code that users can read, inspect, and modify, instead of a black-box program.
We improve on previous solutions by supporting non-commutative DSP filters with a new structural synthesis algorithm.
Future work may include composing filters in a directed graph structure, which would better enable time-varying parameters, and defining a true metric space for distance between audio files.
Other possible improvements could include informing the program of which filters commute, increasing performance where two different structures have the same effect on audio processing, and automatic removal or replacement of filters whose effect is non-zero but trivial (e.g., $PitchShift$ with a small $app$).
Future work may also include a user study to determine if \ourTool proves beneficial to novice DSP programmers.


\begin{acknowledgments}
This research sponsored by NSF grants CCF-1302327 and CCF-1715387.
This research also sponsored by \markk{whoever OMI got money from at Yale}
\end{acknowledgments} 

%%%%%%%%%%%%%%%%%%%%%%%%%%%%%%%%%%%%%%%%%%%%%%%%%%%%%%%%%%%%%%%%%%%%%%%%%%%%%
%bibliography here
\bibliography{icmc2019template}

\end{document}
