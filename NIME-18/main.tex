% Template for NIME 2018
%
% Modified by Luke Dahl on 17 October 2-17
% Modified by Cumhur Erkut on <2016-10-11 Tue>
% Modified by Edgar Berdahl on 5 November 2014
% Modified by Baptiste Caramiaux on 25 November 2013
% Modified by Kyogu Lee on 7 October 2012
% Modified by Georg Essl on 7 November 2011
%
% Based on "sig-alternate.tex" V1.9 April 2009
% This file should be compiled with "nime-alternate.cls"


\documentclass{nime-alternate}



\usepackage{xspace}
\usepackage{caption}
\usepackage{subcaption}


\newcommand{\synthFilter}{\ensuremath{\mathcal{F}}\xspace}


\begin{document}
%
% --- Author Metadata here ---
%\conferenceinfo{NIME'17,}{May 15-19, 2017, Aalborg University Copenhagen, Denmark.}
\conferenceinfo{NIME'18,}{June 3-6, 2018, Blacksburg, Virginia, USA.}

\title{Synthesizing DSP Filters with Programming By Example}


%
% You need the command \numberofauthors to handle the 'placement
% and alignment' of the authors beneath the title.
%
% For aesthetic reasons, we recommend 'three authors at a time'
% i.e. three 'name/affiliation blocks' be placed beneath the title.
%
% NOTE: You are NOT restricted in how many 'rows' of
% "name/affiliations" may appear. We just ask that you restrict
% the number of 'columns' to three.
%
% Because of the available 'opening page real-estate'
% we ask you to refrain from putting more than six authors
% (two rows with three columns) beneath the article title.
% More than six makes the first-page appear very cluttered indeed.
%
% Use the \alignauthor commands to handle the names
% and affiliations for an 'aesthetic maximum' of six authors.
% Add names, affiliations, addresses for
% the seventh etc. author(s) as the argument for the
% \additionalauthors command.
% These 'additional authors' will be output/set for you
% without further effort on your part as the last section in
% the body of your article BEFORE References or any Appendices.

\numberofauthors{4} %  in this sample file, there are a *total*
% of EIGHT authors. SIX appear on the 'first-page' (for formatting
% reasons) and the remaining two appear in the \additionalauthors section.
%
\author{
% You can go ahead and credit any number of authors here,
% e.g. one 'row of three' or two rows (consisting of one row of three
% and a second row of one, two or three).
%
% The command \alignauthor (no curly braces needed) should
% precede each author name, affiliation/snail-mail address and
% e-mail address. Additionally, tag each line of
% affiliation/address with \affaddr, and tag the
% e-mail address with \email.
%
% 1st. author
\alignauthor
Mark Santolucito\\
       \affaddr{Yale University}\\
       \affaddr{51 Prospect St.}\\
       \affaddr{New Haven, CT}\\
       \email{mark.santolucito@yale.edu}
% 2nd. author
\alignauthor
Tom Murphy\\
       \affaddr{Vivid}\\
       \affaddr{P.O. Box 1212}\\
       \affaddr{Dublin, Ohio 43017-6221}\\
       \email{...}
\and
% 3rd. author
\alignauthor Aedan Lombardo\\
       \affaddr{Yale University}\\
       \affaddr{51 Prospect St.}\\
       \affaddr{New Haven, CT}\\
       \email{aedan.lombardo@yale.edu}
%4th
\alignauthor Ruzica Piskac\\
       \affaddr{Yale University}\\
       \affaddr{51 Prospect St.}\\
       \affaddr{New Haven, CT}\\
       \email{ruzica.piskac@yale.edu}
}
% For your initial submission you MUST ANONYMIZE the authors.

\maketitle
%100-200 words
\begin{abstract}
Programming by example allows users to create programs without coding, by simply specifying input and output pairs.
We introduce the problem of digital signal processing programming by example, where users specify input and output wave files, and our tool can automatically synthesize a program that transforms the input to the output.
This program can then be applied to new wave files, giving users a new way to interact with music and program code.
\end{abstract}

\keywords{NIME, proceedings, \LaTeX, template}

% ------- CCS Concepts
% Here is where you enter the CCS Concepts for your paper.
%
% It is strongly recommended that authors view the submission form prior to starting to write the paper, which includes information on the CCS Concepts. 
% 
%The 2012 ACM Computing Classification System (CCS) replaces the traditional 1998 version, which has served as the de facto standard classification system for the computing field. It is being integrated into the search capabilities and visual topic displays of the ACM Digital Library. Please enter the CCS XML code for the classification terms that describe your paper. To get the XML code, please use the following procedure, which is demonstrated using three NIME-related example terms: Applied computing~Sound and music computing, Applied computing~Performing arts, and Information systems~Music retrieval.
%
% 1) Browse to the website http://dl.acm.org/ccs_flat.cfm.
% 2) Select one to three classification terms from the website that describe your paper (e.g. for the example paper Applied computing~Sound and music computing, Applied computing~Performing arts, and Information systems~Music retrieval.).
% 3) For each classification you need to select the relevance (e.g. for this example, Sound and music computing is "high", Performing arts is "low", and Music retrieval is "Medium")
% 4) Once you have selected the last term, click on "view CCS Tex Code". This will generate some code, which includes some CCSXML and some lines beginning with \ccsdesc.
% 5) Keep all of this code, as you will need it for entering into the Precision Conference System paper submission form.
% 6) For this document, keep only the \ccsdesc lines. Here is what you would paste for the classification example:

\ccsdesc[500]{Applied computing~Sound and music computing}
\ccsdesc[100]{Applied computing~Performing arts}
\ccsdesc[300]{Information systems~Music retrieval}

% this line creates the CCS Concepts section.
\printccsdesc


\section{Introduction}

The great proliferation of computer music programming languages points to the difficulty of building a natural interface for users that want to computationally interact with musical data.
Programming applications in the domain of computer music, and specifically digital signal processing (DSP), requires that users not only grasp fundamental programming techniques, but also have a large domain specific knowledge on time and signal manipulations.
The amount of prerequisite skill and effort to overcome these barriers is often higher than many users are able to commit.

Furthermore, the difficult of programming DSP applications is often not commensurate with the creative intentions.
From a musical perspective, take the following simple use-case: a user hears a sample in a piece of music, and later in that piece hears the same sample with some added effects.
Now the user wishes to apply this same effect to their own sample.
In order to recreate this effect on a new sample, the user will have to reprogram the filter from the scratch.
This process will involve writing code, testing the code, listening to the original example, and constantly tweaking parameter values.

To simplify this process, we introduce \textit{DSP programming by example} (DSP-PBE).
In DSP programming by example, a user directly provides an input sound sample and an output sound sample, and a DSP-PBE tool will automatically construct the required program that represents a DSP filter that can transform the input sound into the output sound.

\subsection{Program Synthesis}
Programming by example, and program synthesis more generally, has experienced an explosion of research and progress in the last 15 years within the formal logic research community.
This has led to real world applications, such as FlashFill, a programming by example plugin for Microsoft Excel~\cite{flashfill}.


\section{Motivating Example}

As a motivating example, imagine a user was to reconstruct the filter that was used to transform an audio clip, as shown in Figure~\ref{fig:test}.
In this example, a user provided a clip of a \texttt{cartoon-spring.wav} in Figure~\ref{fig:inEx}, and the same sound as it had been transformed with a low-pass filter at 800 Hz, $lpf(800)$, as shown in Figure~\ref{fig:outEx}.
However the nature of the transformation is unknown to the user and they wish to discover the filter needed.
Our DSP-PBE tool is able to synthesize a filter $lpf(947)$, that when applied to the original sound, produces the waveform shown in Figure~\ref{fig:synthEx}.
While the solution is not exact, the difference is not significantly noticeable to an untrained ear.

\begin{figure}
\centering
\begin{subfigure}{.32\linewidth}
  \centering
  \includegraphics[width=.9\textwidth]{figs/original.png}
  \caption{Input example}
  \label{fig:inEx}
\end{subfigure}%
\begin{subfigure}{.32\linewidth}
  \centering
  \includegraphics[width=.9\textwidth]{figs/lpf800.png}
  \caption{Output example}
  \label{fig:outEx}
\end{subfigure}
\begin{subfigure}{.32\linewidth}
  \centering
  \includegraphics[width=.9\textwidth]{figs/lpf950.png}
  \caption{Generated}
  \label{fig:synthEx}
\end{subfigure}
\caption{The waveforms (a) and (b) are provided as examples, and DSP-PBE synthesizes a filter that produces (c).}
\label{fig:test}
\end{figure}



\section{DSP Synthesis by Example}

Here we define the problem, since this is (as far as I can see) a new problem.


\section{Aural Distance}

As a distance metric, we used as a starting point the literature on acoustic fingerprinting.
Acoustic fingerprinting is the concept of creating a condensed, distinct summary of an audio file that can be used later to identify that audio file or to look it up in a database.
Acoustic fingerprints represent how the file will sound to the human ear regardless of how it is represented in a digital format.
There are numerous ways to develop acoustic fingerprints and companies like Shazam and Sound-Hound have developed complex algorithms to create accurate fingerprints even from low quality files recorded on a cellphone mic.
One of Shazam’s engineers released a paper detailing their process several years ago.
I used this paper as an inspiration for my checker.
The Shazam method uses Fast Fourier Transforms to convert the WAVE file into a spectrogram plotted over time.
Then they plot the (frequency,time) coordinates with the greatest amplitude to create a star map of points that become the files audio fingerprint.
I use a similar strategy by first performing a real Fast Fourier Transform on the imported WAVE file and then picking out the frequency peaks in each time frame.


Fast Fourier Transforms are the key to a good acoustic fingerprint.
Prior to this project, I had only heard of Fourier Transforms in passing and had no idea what they did or how they worked.
I was able to find several Haskell libraries that provide DFT and FFT functions but the majority of my reading on this topic was on how to interpret the results of these functions.
The results of these functions are not presented in an intuitive manner and one must take into account negative frequencies.
The return arrays are presented with the zero frequency component or DC, followed by positive frequencies up until the Nyquist frequency (sampling frequency divided by 2) after which all the elements represent negative frequencies which are irrelevant to a spectrogram.
For this reason, I had to adjust the checker to only observe the array elements from to the Nyquist frequency and exclude the zeroth element and all the negative frequency.

In my readings about the results of FFT I came across the topic of spectral leakage.
Frequency and time are continuous spectra.
FFT, however, relies on a discrete representation of the sound wave over time and therefore returns a discrete representation of the frequency spectrum.
Each return element is a frequency ”bin”, and depending on the scale of your return array the size of this bins varies.
In order for each bin to correspond to 1 Hz the size of the return vector must be equal to the sampling frequency (44,100 Hz).
If each bin is not 1 Hz, you can expect to see the effects of spectral leakage.
This occurs when the bins do not correspond to the exact frequency peaks of the sound.
The amplitude from the peaks that fall in between bins will ”leak” over into the closest bin and create a distorted spectrogram.
For this reason I had to adjust the size of the FFT return arrays to be 44,100.
Although this slows down the process of FFT, it provides the most accurate representation of the sound and for our purposes frequency accuracy is paramount.



\section{Search}
\label{sec:search}

As the search space of possible DSP program is extremely large, our search procedures must be exceptionally efficient. 
As a first foray into DSP-PBE, we restrict ourselves to only synthesizing low-pass and high-pass filters, and global volume adjustment.
These two filters have the key property that they are quasi-commutative - when the thresholds of these filters do not overlap, applying a low-pass and then a high-pass is the same as applying a high-pass and then a low-pass.
We leave the exploration of non-commutative filters (for example, delay lines or ring filters) to future work.

\subsection{Gradient Descent}

Gradient descent is a technique commonly used in modelling and machine learning.
Given a cost function, which represents the disagreement between a proposed model and the actual data, gradient descent can be used efficiently to minimize the cost and generate the model of best fit.
One key restriction on the cost function is that it must be convex - this is because gradient descent will ``descend'' along the surface of the cost function, in each step following the steepest gradient.
We have carefully designed our aural distance function from Section~\ref{sec:distance} to be as close to convex as possible.

\begin{figure}[!h]
\includegraphics[width=\columnwidth]{figs/distCurves} 
\caption{The distance curves showing the convex-like shape of the aural distance function. Each curve is the $dist(\synthFilter(I),O)$, with $\synthFilter = lpf(x)$ where the value of $x$ is taken from the x-axis.}
\label{fig:distCurves}
\end{figure}

In order to visualize the convexity of our distance metric, we plot the distance between pairs of examples, and various possible DSP filters in Figure~\ref{fig:distCurves}.
Here we only visualize the distance curves in the dimension of the low-pass filter.
Notice that the curves exhibit a clear ``saddle'', which represents the minimum cost.
In the ideal case, gradient descent will find these points.
Note that we do not have these graphs available during synthesis - producing the entire graph as in Figure~\ref{fig:distCurves} is prohibitively expensive.

In Figure~\ref{fig:distCurves}, the last curve we plot is the distance between \texttt{cartoon-spring.wav} and \texttt{cartoon-spring-hpf1500.wav}, the same file with a high pass filter applied with a threshold of 1500 Hz.
Notice that as the threshold of the low-pass filter applied to the input example (\texttt{cartoon-spring.wav} increases, the distance to the output example decreases. 
This is because as a low-pass filter's threshold increases, it allows more and more frequencies to pass into the output - thereby having less of an effect.
Whereas in the case of the \texttt{cartoon-spring-hpf1500.wav}, the true filter is a high-pass filter, so the less we apply a low-pass filter, the closer we get to the correct filter.

\begin{figure}[!h]
\includegraphics[width=\columnwidth]{figs/distCurveZoom} 
\caption{Zooming in (1000 to 1500 Hz) on a portion of a curve from Figure~\ref{fig:distCurves}, we see the aural distance function is not perfectly convex on the micro scale.}
\label{fig:microDist}
\end{figure}

There are a number challenges with working with gradient descent in the aural DSP domain.
The first is that the domain can have a high number of dimensions to search.
In our implementation, we only explore a two DSP filters and volume adjustment, but this results in 5 dimensional space (each filter requires both a threshold value and an amplitude value for how much of the filter to apply).
To speed up gradient descent, we use stochastic gradient descent, so that in each step, we only move in $d<5$ number of dimensions.

Additionally, on the micro scale, the distance function is susceptible to noise and not entirely smooth, as shown in Figure~\ref{fig:microDist}.
In order to handle the micro scale variations, we use a periodic restart of the gradient descent.
This means that every $n$ rounds, as defined by the user (we use 4), the gradient descent will backtrack to the best solution it has found so far.
The stochastic gradient descent will then continue, selecting dimensions to explore in each round using a new random seed.

\begin{figure}[!h]
\includegraphics[width=\columnwidth]{figs/distCurveMacro} 
\caption{Looking at the portion of a curve from Figure~\ref{fig:distCurves} between 8k Hz and 20k Hz, we see the aural distance function is not perfectly convex on the macro scale. In this case, that is because the sample has very few frequencies above the 8k Hz range.}
\label{fig:macroDist}
\end{figure}

On the macro scale, we face the challenge that the distance function is only quasi-convex - there are many local minimum and long plateaus, as shown in Figure~\ref{fig:macroDist}.
In order to overcome this, we must carefully pick the initial value for gradient descent.
If we pick a value in the middle of a plateau, the gradient descent algorithm will not find any significant gradient, and conclude we have reached the convergence condition.
In our current implementation, we iterate at large intervals (1000 Hz) possible threshold values for both low and high pass filters.
We choose possible DSP programs that use only low pass, only high pass, and both low and high pass filters.
After evaluating these, we take the lowest cost initial DSP program, and start gradient descent from that point.

Finally, one of the key parts of a good application of gradient descent is the choice of the parameters such as the learning rate and the convergence goal.
These parameters must be adjusted based on the values observed from the cost (in our case, distance) function.
While the details of tuning gradient descent are outside the scope of this paper, it suffices to note that any change in the distance metric will likely also require a readjustment of these parameters.



\section{Evaluation}
\label{sec:eval}
\begin{table*}[!ht]
\centering
\setlength\tabcolsep{2em}
\begin{tabular}{|c | c | c | c |} 
 \hline
 Description & True DSP & Synth'ed DSP & Time (sec) \\ 
 \hline\hline
 Cartoon Spring & $lpf(800)  $ & $lpf(1989) $ & 56.195 \\
 Cartoon Spring & $lpf(5000) $ & $lpf(4000) \arrComp hpf(7000) $ & 54.004 \\
 Cartoon Spring & $hpf(1500) $ & $lpf(1000) \arrComp hpf(1000) $ & 53.964 \\
 BTS DNA (Kpop) & $lpf(2000) $ & $lpf(1996)	 $ & 56.874 \\
 Holst Mars     & $hpf(3500) $ & $lpf(10000) \arrComp hpf(1000)$  & 55.444 \\
 \hline
\end{tabular}
\caption{Time to converged on a solution DSP program for various benchmarks. The programs may not match the known DSP program, but may still be psycho-acoustically equivalent depending on the expertise of the listener. }
\label{table:eval}
\end{table*}

We implemented a DSP-PBE tool based on the approach described in Section~\ref{sec:distance} and Section~\ref{sec:search}.
Our tool is available open-source at~\url{www.github.com/santolucito/DSP-PBE}.
Our tool is mostly written in Haskell and uses the Vivid library~\cite{vivid} for bindings to SuperCollider~\cite{supercollider}.
Haskell allows easy access to type information and metaprogram construction tools that are useful for program synthesis, however the programs themselves are easily translated back to SuperCollider ``synth defs'', which are DSP filter programs.
We use the scipy python module for calling the FFT since the library is quite mature and provides a simplified interface specifically for calling FFT on audio.

One key implementation point is that we use a separate representation of a DSP for running gradient descent, and for actually processing the audio.
Gradient descent works best when all parameters are in the same scale, so we map the frequencies [0,20k] Hz to a [-1,1] scale.
Likewise, we map the application levels for each filter (how much of the filtered output should be included in the final mix) on a [-1,1] scale.

In Table~\ref{table:eval}, we show the results of running our tool on a set of benchmarks of input/output example audio samples.
The audio samples were transformed in Audacity, using the Low Pass Filter and High Pass Filter effects.
Since we use SuperCollider's filter implementations on the backend, there may be very slight variation, but this is to be expected in real-world application as well.
All experiments were run on an Intel Core i7-6820HQ CPU @ 2.70GHz with 16 GB of RAM and an Intel Sunrise Point-H HD Audio sound card.


We can also breakdown the runtime cost of synthesis into the two different stages - 1) initial program selection, and 2) gradient descent.
The initial program selection phase is a mostly fixed cost, as we always evaluate the same distribution of initial value.
On average this process takes roughly 40 seconds.
We outline future directions of research that may be able to reduce this cost in Section~\ref{sec:rtypes}.


\section{Future Work and Conclusions}

The main contribution of this paper is to pose the problem of DSP-PBE.
While we have presented a prototype implementation of a DSP-PBE tool, this primarily functions as a proof-of-concept.
There remains significant room for optimization in both the distance calculation and the search algorithm.

Although with the current tool, synthesis times presented might be prohibitively slow for many use cases, we should be encouraged by progress in other domains of program synthesis.
In the SyGuS program synthesis competition, which has run for 4 years, tools have seen an exponential speed up and increase in the range of programs that can be synthesized.
As one example, in the 2014 competition the \texttt{LinExpr\_eq1.sl} benchmark was only solved by one tool, and took 1128 seconds in 2014~\cite{sygus2014}.
In the 2017 competition, the same benchmark was solved by all tools, with the fastest taking only 199 seconds~\cite{sygus2017}.


%ACKNOWLEDGMENTS are optional
\section{Acknowledgments}
This work was supported the NSF grant XXXXXXXX and by viewers like you.

%
% The following two commands are all you need in the
% initial runs of your .tex file to
% produce the bibliography for the citations in your paper.
\bibliographystyle{abbrv}
\bibliography{nime-references}  % sigproc.bib is the name of the Bibliography in this case
% You must have a proper ".bib" file
%  and remember to run:
% latex bibtex latex latex
% to resolve all references
%
% ACM needs 'a single self-contained file'!
%

%%% Place this command where you want to balance the columns on the last page. 
%\balancecolumns 

% That's all folks!
\end{document}
